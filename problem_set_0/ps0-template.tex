%
% 6.006 problem set 0 solutions template
%
\documentclass[12pt,twoside]{article}

\input{macros-fa19}
\newcommand{\theproblemsetnum}{0}
\newcommand{\releasedate}{Thursday, September 5}
\newcommand{\partaduedate}{Sunday, September 8}
\newcommand{\probP}{\text{I\kern-0.15em P}}

\usepackage{relsize}
\usepackage{amsmath}

\title{6.006 Problem Set 0}

\begin{document}

\handout{Problem Set \theproblemsetnum}{\releasedate}
\textbf{All parts are due {\bf \partaduedate} at {\bf 6PM}}.

\setlength{\parindent}{0pt}
\medskip\hrulefill\medskip

{\bf Name:} Your Name

\medskip\hrulefill


%%%%%%%%%%%%%%%%%%%%%%%%%%%%%%%%%%%%%%%%%%%%%%%%%%%%%
% See below for common and useful latex constructs. %
%%%%%%%%%%%%%%%%%%%%%%%%%%%%%%%%%%%%%%%%%%%%%%%%%%%%%

% Some useful commands:
% $f(x) = \Theta(x)$
% $T(x, y) \leq \log(x) + 2^y + \binom{2n}{n}$
% \ttt{code\_function}


% You can create unnumbered lists as follows:
% \begin{itemize}
%     \item First item in a list 
%         \begin{itemize}
%             \item First item in a list 
%                 \begin{itemize}
%                     \item First item in a list 
%                     \item Second item in a list 
%                 \end{itemize}
%             \item Second item in a list 
%         \end{itemize}
%     \item Second item in a list 
% \end{itemize}

% You can create numbered lists as follows:
% \begin{enumerate}
%     \item First item in a list 
%     \item Second item in a list 
%     \item Third item in a list
% \end{enumerate}

% You can write aligned equations as follows:
% \begin{align} 
%     \begin{split}
%         (x+y)^3 &= (x+y)^2(x+y) \\
%                 &= (x^2+2xy+y^2)(x+y) \\
%                 &= (x^3+2x^2y+xy^2) + (x^2y+2xy^2+y^3) \\
%                 &= x^3+3x^2y+3xy^2+y^3
%     \end{split}                                 
% \end{align}

% You can create grids/matrices as follows:
% \begin{align}
%     A = 
%     \begin{bmatrix}
%         A_{11} & A_{21} \\
%         A_{21} & A_{22}
%     \end{bmatrix}
% \end{align}

\begin{problems}

\problem  % Problem 1
$A=\{1,2,4,8,16\}$ and $B=\{-1,1,3,5\}$
\begin{problemparts}
\problempart % Problem 1a
$A \cap B = \{1\}$
\problempart % Problem 1b
$\lvert A \cup B \rvert = \lvert \{1,2,4,8,5,3,16,-1\} \rvert=8$
\problempart % Problem 1c
$\lvert A - B \rvert = \lvert \{2,4,8,16\} \rvert=4$
\end{problemparts}

\problem  % Problem 2
\begin{problemparts}
\problempart % Problem 2a
$ \probP(0)=\dfrac{2}{5} * \dfrac{1}{4} = \dfrac{1}{10} $ \\ [1ex]
$ \probP(1)=\dfrac{3}{5}$ \\ [1ex]
$ \probP(2)=\dfrac{3}{5}*\dfrac{2}{4}=\dfrac{3}{10}$ \\ [1ex]
\begin{tabular}{|c c c c|} 
 \hline
 X & 0 & 1 & 2 \\ [0.5ex] 
 \hline
 \probP & ${\dfrac{1}{10}}$ & ${\dfrac{6}{10}}$ & ${\dfrac{3}{10}}$ \\ [1ex]
 \hline
\end{tabular} \\

$\mathbb{E}[X]=\sum_{i=1}^{3} x_i p_i = 0*0.1 + 1*0.6 + 2*0.3 = 1.2$ \\
\problempart % Problem 2b
$ $ \\
\begin{tabular}{|c c c c|} 
 \hline
 Y & 0 & 1 & 2 \\ [0.5ex] 
 \hline
 \probP & ${\dfrac{1}{4}}$ & ${\dfrac{1}{2}}$ & ${\dfrac{1}{4}}$ \\ [1ex]
 \hline
\end{tabular} \\ [1em]
$\mathbb{E}[Y]=1$
\problempart % Problem 2c
$\mathbb{E}[X+Y]=\mathbb{E}[X]+\mathbb{E}[Y]=1.2+1=2.2$
\end{problemparts}

\problem  % Problem 3

\begin{problemparts}
\problempart % Problem 3a
$A \equiv B \mod 2$ \\ [1em]
$(606 - 360) \mod 2 = 0 \Rightarrow true$ \\
\problempart % Problem 3b
$A \equiv B \mod 3$ \\ [1em]
$(606 - 360) \mod 3 = 0 \Rightarrow true$ \\
\problempart % Problem 3c
$A \equiv B \mod 4$ \\ [1em]
$(606 - 360) \mod 4 = 2 \Rightarrow false$ \\
\end{problemparts}

\problem  % Problem 4
Base case:
$n=0 \text{ then } \mathlarger{\mathlarger{\sum}}_{i=0}^{0}a^{i}=\frac{1-a^{0}}{1-a} = 1$ Assume that is true for $n=k$. We want to show that the statement is true for $n=k+1$.

$\mathlarger{\mathlarger{\sum}}_{i=0}^{k+1}a^{i}=\frac{1-a^{k+2}}{1-a}$ \\
$\mathlarger{\mathlarger{\sum}}_{i=0}^{k+1}a^{i}=a^{k+1}+\mathlarger{\sum}_{i=0}^{k}a^{i}=a^{k}+\frac{1-a^{k+1}}{1-a}=\frac{a^{k+1}(1-a)}{1-a}+\frac{1-a^{k+1}}{1-a}=\frac{a^{k+1}(1-a)+(1-a^{k+1})}{1-a}=\frac{a^{k+1}-a^{k+2}+1-a^{k+1}}{1-a}=\frac{1-a^{k+2}}{1-a}$, as desired. \\

\problem  % Problem 5
Base case: A tree contains only one vertex without edges. It means it can be colored to red or blue color. Assume that is true for k vertexes of the tree. We want to show that the statement is true for $k+1$ vertexes. From the condition, we have that the vertex $v$ connected with only one edge with vertex $u$. If we delete the edge which connects $v$
 and $u$ edges, we will have the $T'$ tree. The $T'$ tree contains $k$ vertexes. By the induction hypothesis, the tree consisting $k$ vertexes doesn't have the adjacent vertices with the same color. Thus, the color of $u$ vertex must be colored to opposite color than vertex $v$. Therefore, anyone edge in the tree doesn't connect vertexes with the same color.
 
\vfill

\end{problems}

\end{document}